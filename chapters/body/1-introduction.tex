\chapter{Introductie}

\emph{maximaal 800 woorden}

\section*{Aanleiding}

In een tijd waarin data-intelligentie en geavanceerde analysetechnieken een steeds grotere rol spelen in zowel de academische wereld als de industrie, wordt het cruciaal voor organisaties om effectief gebruik te maken van hun beschikbare gegevens. Het Lectoraat Data Intelligence van Hogeschool Zuyd staat bekend om zijn toewijding aan onderzoek en innovatie op het gebied van data-intelligentie, met een specifieke focus op het ontwikkelen van datagedreven oplossingen voor maatschappelijke vraagstukken in Limburg en daarbuiten. 

Het lectoraat staat voor de uitdaging om zijn proces van projectregistratie en -beheer te optimaliseren. Handmatige procedures voor het vastleggen van projectinformatie, het bijhouden van voortgang en het presenteren van resultaten hebben geleid tot inefficiëntie en belemmeren het vermogen van het lectoraat om zijn prestaties effectief te communiceren en te verantwoorden. Bovendien belemmert het gebrek aan een gestroomlijnd systeem de traceerbaarheid en presentatie van projecten, wat resulteert in een behoefte aan een geautomatiseerde oplossing die deze processen vereenvoudigt en verbetert.

Deze uitdagingen hebben geleid tot het initiatief om een geautomatiseerd systeem te ontwikkelen voor het beheren en visualiseren van projectgegevens binnen het Lectoraat Data Intelligence. Dit project heeft tot doel het huidige proces van projectregistratie en -beheer te verbeteren door middel van innovatieve technologieën en methodologieën, en zo de efficiëntie en effectiviteit van het lectoraat te vergroten. Dit extended abstract biedt een overzicht van de aanpak, methodologie en resultaten van dit onderzoeksproject, met als uiteindelijke doel het presenteren van een praktische oplossing die de operationele processen van het lectoraat transformeert en versterkt.


\section*{Doelstelling}

